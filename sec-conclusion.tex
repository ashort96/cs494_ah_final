%!TEX root = ./main.tex



\section{Conclusion \& Future Work}



Developing with low level languages, such as RISC-V, is difficult.
%
There are not many tools available, especially to students, that help them understand what is going on under the hood.
%
The \tool{} aims to help its users to understand what each line of code is doing and how it interacts with the computer.
%


\tool{} could be improved several ways.
%
For one, it could include more useful and helpful error messages.
%
Currently, the tool only gives very general error messages, such as it cannot read value X at line Y.
%
The reason for that is that the code is not actually compiled.
%
Instead, a parser and lexer was written for the application.
%
Having the code actually be compiled in a sandbox would be more accurate, and would give better feedback.
%
Additionally, it would require less updates on our end as the language continues to develop.
%
If we have a new backend be created that can execute code in a sandboxed environment, this will lessen the stress on the client and make it easier to continue the development.
%


Overall, I believe this tool can help novice programmers grasp of how RISC-V and other assembly languages affect the machine as it is being ran.