%!TEX root = ./main.tex



\section{Proposed Evaluation}
To evaluate the tool, students will be picked from the CS130 course at the University of Tennessee, Knoxville.
%
These students will be within their first year of Computer Science, and they will be learning RISC-V.
%
Assignments will be uploaded to the application, and students will be asked to complete the tasks within a fifty minute time frame.
%
There will be two tasks, one focusing on the debugging feature and the other focusing on the adaptability of creating assignments.


\subsection{Testing Debugging}
To test the debugging features, all students will start an assignment that has code available for them.
%
The code will have an error that will still allow the code to compile, but will cause a value to be incorrect somewhere in the middle of the code.
%
The students will be expected to not only identify where the issue is, but also fix the error.
%
Half the class will be allowed to use the tool, while the other half must only use the tools available to them on the machines at the university.
%
The students will not be permitted to work together.
%
The times of how long it takes the task will be recorded.
%
In addition to this test, another test will be given where the code never ends execution.
%
Both groups will also be asked to identify where the infinite loop is coming from and to fix it.
%
The times of this will be recorded similarly to the previous test.


\subsection{Testing Assignments}
An assignment will be created, but no code will be given with the assignment.
%
Registers will have certain values and the participants will be asked to complete a task based off of these values.
%
This will allow us to determine if the layout is easy to use and information is easily accessible.
%
The goal of this exercise is not to determine the ability of the student's programming, but rather how well the design works.
%
An additional assignment will be created that will test how students use the tool to write code that accomplishes a certain task.
%
They will be given a problem that, for their level of programming, will be slightly difficult, but doable within a thirty minute time span.
%
